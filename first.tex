%% It is just an empty TeX file.
%% Write your code here.
\documentclass[a4paper,12pt]{article}
\begin{document}
\title{Homework 1 for ECS 20}
\author{Tamim Nekaien (915803826)}
\date{\today}
\maketitle

This assignment was made with 
$\heartsuit$ (and \LaTeX).

\pagenumbering{roman}
\tableofcontents
\newpage
 
\section{Question 1}
\label{sec1}

\begin{center}
  \begin{tabular}{ |c|c|c|c|c|c|c|c| }
    \hline
    p & q & r & $ \neg q$ & $\neg r$ & $r \Rightarrow \neg q$ & $p \wedge \neg r$ & $\neg(r \Rightarrow \neg q) \vee (p \wedge \neg r)$  \\ \hline
    1 & 1 & 1 & 0 & 0 & 0 & 0 & 1 \\ \hline
    1 & 1 & 0 & 1 & 0 & 1 & 1 & 1 \\ \hline
    1 & 0 & 1 & 0 & 1 & 1 & 0 & 0 \\ \hline
    1 & 0 & 0 & 1 & 1 & 1 & 1 & 1 \\ \hline
    0 & 1 & 1 & 0 & 0 & 0 & 0 & 1 \\ \hline
    0 & 1 & 0 & 1 & 0 & 1 & 0 & 0 \\ \hline
    0 & 0 & 1 & 0 & 1 & 1 & 0 & 0 \\ \hline
    0 & 0 & 0 & 1 & 1 & 1 & 0 & 0 \\ 
    \hline
  \end{tabular}
\end{center}


\section{Question 2}
\label{sec2}

\subsection{part 1}
\label{sec3}

\begin{center}
  \begin{tabular}{ |c|c|c|c|c|c|c| }
    \hline
    p & q & r & $p \Rightarrow q$ & $(p \Rightarrow q) \Rightarrow r$ & $q \Rightarrow r$ & $p \Rightarrow (q \Rightarrow r)$  \\ \hline
    1 & 1 & 1 & 1 & 1 & 1 & 1 \\ \hline
    1 & 1 & 0 & 1 & 0 & 0 & 0 \\ \hline
    1 & 0 & 1 & 0 & 1 & 1 & 1 \\ \hline
    1 & 0 & 0 & 0 & 1 & 1 & 1 \\ \hline
    0 & 1 & 1 & 1 & 1 & 1 & 1 \\ \hline
    0 & 1 & 0 & 1 & 0 & 0 & 1 \\ \hline
    0 & 0 & 1 & 1 & 1 & 1 & 1 \\ \hline
    0 & 0 & 0 & 1 & 0 & 1 & 1 \\ 
    \hline
  \end{tabular}
\end{center}

The two statements are not equal.
\subsection{part 2}
\label{sec4}
\begin{enumerate}
    \item $(q \wedge (p \Rightarrow \neg q)) \Rightarrow \neg p \longrightarrow $ Original statement.
    \item $(q \wedge (\neg p \vee \neg q)) \Rightarrow \neg p \longrightarrow $ Implication into not and or.
    \item $(q \wedge \neg p) \vee (q \wedge \neg q) \Rightarrow \neg p \longrightarrow$ Distributive property.
    \item $(q \wedge \neg p) \vee (F) \Rightarrow \neg p \longrightarrow$ Converting to falsehoods.
    \item $(q \wedge \neg p) \Rightarrow \neg p \longrightarrow$ False or statement is just statement.
    \item $\neg(q \wedge \neg p) \vee \neg p \longrightarrow$ Implication into not and or.
    \item $(\neg q \vee p) \vee \neg p \longrightarrow$ De Morgan's law.
    \item $\neg q \vee (p \vee \neg p) \longrightarrow$ Associative Property.
    \item $\neg q \vee (T) \longrightarrow$ Converting to Truth
    \item $(T) \longrightarrow$ Truth or anything is just truth.
\end{enumerate}

\section{Question 3}
\label{sec5}

\subsection{part 1}
\label{sec6}
As an implication: If the weather changed very rapidly, then a tornado would have touched down on Davis last week.
\subsection{part 2}
\label{sec7}
As a converse: If a tornado touched down on Davis last week, then the weather has changed very rapidly.
\subsection{part 3}
\label{sec9}
As a contrapositive: If a tornado didn't touch down on Davis last week, then the weather isn't changing rapidly.
\section{Question 4}
\label{sec8}

\begin{center}
  \begin{tabular}{ |c|c|c|c|c| }
    \hline
    A & B & B is a knave & One is a knight & Consistent?  \\ \hline
    Knight & Knave & True & True & Not possible  \\ \hline
    Knave & Knight & False & True & Possible  \\ 
    \hline
  \end{tabular}
\end{center}
A is a Knave and B is a Knight.
\section{Answers}
My answer for \ref{sec1} , \ref{sec2} , \ref{sec3} , \ref{sec4} , \ref{sec5} , \ref{sec6} , \ref{sec7} ,\ref{sec9}, \ref{sec8} are on pages \pageref{sec1} and \pageref{sec7}.


\end{document}