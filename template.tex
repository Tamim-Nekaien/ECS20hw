% This is a template resource for students that want to start right away with LaTeX with a simple guiding template
% links for resources (copy and paste in search bar): 

% General coverage of everything: http://mirrors.ibiblio.org/CTAN/info/lshort/english/lshort.pdf
% More on Tables:  https://en.wikibooks.org/wiki/LaTeX/Tables
% Math symbols: https://oeis.org/wiki/List_of_LaTeX_mathematical_symbols
% Editor: http://papeeria.com/
% Editor: https://www.overleaf.com/

% Note: the \ signals to Tex that there is an element being used, the $ is put before and after math elements to signify math type.

\documentclass[a4paper,12pt]{article} % signifies type of document (no need to change this)
\usepackage{newtxtext} % TEx has it's own library, include as many packages as you want here
\usepackage{newtxmath}

\begin{document} % start of a document (no need to change this)

\title{Homework \# for ECS 20} % Title (change the #)
\author{Joe Kerr (ID:)} % Name of "homework-doer" (change the name and number)
\date{\today} % start of a document (no need to change this)
\maketitle % tells LaTeX to do everything above.

This assignment was made with \LaTeX. % Leave this here

\pagenumbering{roman} % You can change this to arabic if you want
\tableofcontents % This creates a table of contents for your homework problems

\newpage % Basically page break
 
\section{Question 1} % Organize homework questions into sections
\label{sec1} % Label and name it to reference for later

\begin{center} % We are creating a table now
  \begin{tabular}{ |c|c|c|c| } % the c stands for center of cell, the | serves as a colunn divider
    \hline % row divider
    p & q & r & $ \neg q$ \\ \hline 
    T & T & T & F  \\ \hline % The & signals to Tex to move on to next column
    T & T & F & F  \\ \hline
    T & F & T & T  \\ \hline
    T & F & F & T  \\ \hline % Use ctrl+click multiple lines to edit multiple lines at once
    F & T & T & F  \\ \hline
    F & T & F & F  \\ \hline
    F & F & T & T  \\ \hline
    F & F & F & T  \\ 
    \hline
  \end{tabular} % Closes everything off
\end{center}


\section{Question 2} 
\label{sec2}

\subsection{part 1} % This is in case you get a two-part Question.
\label{sec3}

\subsection{part 2}
\label{sec4}
\begin{enumerate} % Creates numbered list for proofs and algebra steps for a problem.
    \item $(q \wedge (p \Rightarrow \neg q)) \Rightarrow \neg p \longrightarrow $ Original statement. % item is a new list element
    \item $(T) \longrightarrow$ Truth or (something) is Truth.
\end{enumerate}

\section{Question 3}
\label{sec5}

\subsection{part 1}
\label{sec6}
Modus Ponens
 $\begin{array}{rl} % arrays for modus ponens
    & p \to q\\
    & p \\
    \cline{2-2}
    \therefore & q
  \end{array}$

\subsection{part 2}
\label{sec7}

\subsection{part 3}
\label{sec9}

\section{Question 4}
\label{sec8}


\section{Answers}
My answer for \ref{sec1} , \ref{sec2} , \ref{sec3} , \ref{sec4} , \ref{sec5} , \ref{sec6} , \ref{sec7} ,\ref{sec9}, \ref{sec8} is on page \pageref{sec1}. % This is how you reference the number that the questions are and page numbers too


\end{document}